\documentclass[conference]{IEEEtran}
\usepackage{graphicx}
\usepackage{listings}
\usepackage{url}
\usepackage{hyperref}

\title{Ainara: An Open Source Framework for Building Modular AI Assistants}

\author{
    \IEEEauthorblockN{Rubén Gómez Agudo}
    \IEEEauthorblockA{Khromalabs et al.\\
    Email: ruben@khromalabs.org}
}

\begin{document}
\maketitle

\begin{abstract}
This paper presents Ainara, an open-source framework for building modular and extensible AI assistants. The framework introduces a novel architecture that combines large language models with specialized skill modules and configurable recipes, enabling the creation of versatile AI assistants. We describe the core components, including the capabilities manager, skill system, and text-to-speech integration, demonstrating how this modular approach enhances both functionality and maintainability.
\end{abstract}

\section{Introduction}
The development of AI assistants has traditionally followed monolithic architectures, limiting their extensibility and adaptability. Ainara addresses these limitations by introducing a modular framework that separates core functionality into distinct, reusable components. This approach not only facilitates maintenance and testing but also allows for rapid integration of new capabilities.

\section{Architecture}
\subsection{Core Components}
The framework consists of several key components:
\begin{itemize}
    \item Capabilities Manager: Handles dynamic registration and management of skills and recipes
    \item LLM Backend: Provides abstraction for different language model providers
    \item Skill System: Enables modular extension of assistant capabilities
    \item Recipe System: Allows creation of complex workflows combining multiple skills
\end{itemize}

\subsection{Skill System}
Skills in Ainara are self-contained modules that implement specific functionality. The framework provides base classes and interfaces for creating new skills, ensuring consistency and maintainability. Examples include:
\begin{itemize}
    \item HTML processing skills (download, distillation)
    \item News search and analysis
    \item Sentiment analysis across multiple platforms
    \item Text-to-speech capabilities
\end{itemize}

\section{Implementation}
\subsection{Capabilities Management}
The CapabilitiesManager class serves as the central coordinator for all available functionality. It dynamically loads skills and recipes, manages their lifecycle, and provides a unified interface for accessing them.

\subsection{Language Model Integration}
The framework implements an abstract LLMBackend class that allows for easy integration of different language model providers. This abstraction enables:
\begin{itemize}
    \item Seamless switching between different LLM providers
    \item Standardized interface for text processing
    \item Support for streaming responses
\end{itemize}

\section{Text-to-Speech Integration}
A notable feature is the framework's sophisticated text-to-speech system, which provides:
\begin{itemize}
    \item Synchronized text display with audio playback
    \item Natural phrase splitting for improved readability
    \item Fallback mechanisms for robustness
\end{itemize}

\section{Results and Discussion}
[This section will discuss performance metrics, use cases, and comparative analysis]

\section{Future Work}
Potential areas for future development include:
\begin{itemize}
    \item Enhanced multi-modal capabilities
    \item Improved recipe composition tools
    \item Extended skill marketplace
    \item Advanced caching mechanisms
\end{itemize}

\section{Conclusion}
Ainara demonstrates the effectiveness of a modular approach to building AI assistants. The framework's architecture provides a solid foundation for creating sophisticated AI applications while maintaining flexibility and extensibility.

\bibliographystyle{IEEEtran}
\begin{thebibliography}{1}
\bibitem{gpt3} Brown, T. B., et al. "Language models are few-shot learners." arXiv preprint arXiv:2005.14165 (2020).
\end{thebibliography}

\end{document}
